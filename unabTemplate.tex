\documentclass[aspectratio=169]{beamer}
\usetheme{unab}

\usepackage[T1]{fontenc}
\usepackage[spanish]{babel}

\usepackage{amsmath}
\usepackage{amsfonts}
\usepackage{amssymb}

\usepackage{ulem}
\usepackage{array}
\usepackage{siunitx}

% Configuración de colores para mostrar comandos de teclado
\usepackage[os=win]{menukeys}
\newmenucolortheme{mycolors}{named}{white}{unabburdeo}{unabblue}
\changemenucolortheme{menus}{mycolors}
\changemenucolortheme{roundedkeys}{mycolors}

\hypersetup{colorlinks,linkcolor=,urlcolor=green!50!black}

\graphicspath{{figs/}}
\DeclareGraphicsExtensions{.png, .jpg, .pdf}

%%%%%%%%%%%%%%%%%%%%%%%%%%%%%%%%%%%%%%%%%%%%%%%%%%%%%%%%%%%%%%%%%%%%%%%%
% CONFIGURACIÓN DE CÓDIGO
%%%%%%%%%%%%%%%%%%%%%%%%%%%%%%%%%%%%%%%%%%%%%%%%%%%%%%%%%%%%%%%%%%%%%%%%
\usepackage{listings}

\renewcommand{\lstlistingname}{Código}
\lstset{
	basicstyle=\small\ttfamily,
	extendedchars=\true,
    captionpos=b,
	breaklines=true,
	tabsize=4,
	showstringspaces=false,
	aboveskip=\baselineskip,
	belowskip=1.25\baselineskip,
	extendedchars=\true,
	inputencoding=utf8,
	literate={á}{{\'a}}1
	 {é}{{\'e}}1
	 {í}{{\'i}}1 
	 {ó}{{\'o}}1
	 {ú}{{\'u}}1
	 {ñ}{{\~n}}1
	 {ò}{{\`o}}1
	 {ü}{{\"u}}1,
}

\lstdefinestyle{python}
{
    language={Python},
    frame=single,
    %===========================================================
    framesep=3pt,%expand outward.
    framerule=0.4pt,%expand outward.
    xleftmargin=3.4pt,%make the frame fits in the text area. 
    xrightmargin=3.4pt,%make the frame fits in the text area.
    %=========================================================== 
    rulecolor=\color{unabburdeo},
    %=========================================================== 
    morekeywords={as, self},
    keywordstyle=\color{orange},
    %identifierstyle=\color{cyan},
    stringstyle=\color{red},
    commentstyle=\color{teal},
    %=========================================================== 
    morekeywords={True, False},
}

%%%%%%%%%%%%%%%%%%%%%%%%%%%%%%%%%%%%%%%%%%%%%%%%%%%%%%%%%%%%%%%%%%%%%%%%%%%%%%%%%%%
% Información del documento
\title{Template de Ejemplo}
\subtitle{Usando formato de la \textsc{Unab}}
\date{\today}
\author[Autor]{Le Autor}
\institute{Universidad Andrés Bello}
	

%%%%%%%%%%%%%%%%%%%%%%%%%%%%%%%%%%%%%%%%%%%%%%%%%%%%%%%%%%%%%%%%%%%%%%%%%%%%%%%%%%%
% INICIO
\begin{document}

\AtBeginSection{
	\begin{frame}{Contenidos}
		\tableofcontents[currentsection]
	\end{frame}
}

\begin{frame}
  \titlepage
\end{frame}

\section{Sección random}
\begin{frame}{Título}{Y le subtítulo}
	Le texto de ejemplo.
	
	En varios párrafos y con algo de texto \alert{resaltado}
\end{frame}

\begin{frame}{Le Listas}{Más ejemplos}
	\begin{itemize}
	\item Listas con cosas random:
		\begin{itemize}
		\item Como esta.
		\item Y esta.
		\item Y esta otra.
		\end{itemize}

	\item Para enumerar, también:
		\begin{enumerate}
		\item Le primer ítem
		\item Le segundo ítem
		\end{enumerate}
	\end{itemize}
\end{frame}

\begin{frame}{Le Bloques}{Ejemplos de los más usados}
	\begin{block}{Le título}
		Le bloque con \alert{título} (bloque básico, en realidad).
	\end{block}
	
	\begin{definition}
		Este es para \alert{definir} cosas.
	\end{definition}
	
\end{frame}

\begin{frame}[allowframebreaks]{Le Bloques matemáticos}{Teorema y pruebas}
	\begin{theorem}[De ejemplo]
		El bloque de estilo básico es el \alert{theorem}.
	\end{theorem}

	\begin{proof}
		La demostración es que los demás bloques están basados en ese.
	\end{proof}
	
	\begin{example}[Le ejemplo]
		Como este.
	\end{example}
	
	\begin{corollary}
		Podemos incluir corolarios y porquerías.
	\end{corollary}
	
	\begin{fact}
		Le constatación de hechos
	\end{fact}
	
	\begin{lemma}
		Un lema loco\ldots
	\end{lemma}
\end{frame}

\end{document}
